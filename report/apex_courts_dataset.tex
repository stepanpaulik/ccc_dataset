% ARTICLE 2 ----
% This is just here so I know exactly what I'm looking at in Rstudio when messing with stuff.
% Options for packages loaded elsewhere
\PassOptionsToPackage{unicode}{hyperref}
\PassOptionsToPackage{hyphens}{url}
%
\documentclass[
  11pt,
]{article}
\usepackage{lmodern}
\usepackage{amssymb,amsmath}
\usepackage{ifxetex,ifluatex}
\ifnum 0\ifxetex 1\fi\ifluatex 1\fi=0 % if pdftex
  \usepackage[T1]{fontenc}
  \usepackage[utf8]{inputenc}
  \usepackage{textcomp} % provide euro and other symbols
\else % if luatex or xetex
  \usepackage{unicode-math}
  \defaultfontfeatures{Scale=MatchLowercase}
  \defaultfontfeatures[\rmfamily]{Ligatures=TeX,Scale=1}
\fi
% Use upquote if available, for straight quotes in verbatim environments
\IfFileExists{upquote.sty}{\usepackage{upquote}}{}
\IfFileExists{microtype.sty}{% use microtype if available
  \usepackage[]{microtype}
  \UseMicrotypeSet[protrusion]{basicmath} % disable protrusion for tt fonts
}{}
\makeatletter
\@ifundefined{KOMAClassName}{% if non-KOMA class
  \IfFileExists{parskip.sty}{%
    \usepackage{parskip}
  }{% else
    \setlength{\parindent}{0pt}
    \setlength{\parskip}{6pt plus 2pt minus 1pt}
    }
}{% if KOMA class
  \KOMAoptions{parskip=half}}
\makeatother
\usepackage{xcolor}
\IfFileExists{xurl.sty}{\usepackage{xurl}}{} % add URL line breaks if available
\urlstyle{same} % disable monospaced font for URLs
\usepackage[margin=1in]{geometry}
\setlength{\emergencystretch}{3em} % prevent overfull lines
\providecommand{\tightlist}{%
  \setlength{\itemsep}{0pt}\setlength{\parskip}{0pt}}
\setcounter{secnumdepth}{5}

\ifluatex
  \usepackage{selnolig}  % disable illegal ligatures
\fi
\newlength{\cslhangindent}
\setlength{\cslhangindent}{1.5em}
\newlength{\csllabelwidth}
\setlength{\csllabelwidth}{3em}
\newenvironment{CSLReferences}[2] % #1 hanging-ident, #2 entry spacing
 {% don't indent paragraphs
  \setlength{\parindent}{0pt}
  % turn on hanging indent if param 1 is 1
  \ifodd #1 \everypar{\setlength{\hangindent}{\cslhangindent}}\ignorespaces\fi
  % set entry spacing
  \ifnum #2 > 0
  \setlength{\parskip}{#2\baselineskip}
  \fi
 }%
 {}
\usepackage{calc}
\newcommand{\CSLBlock}[1]{#1\hfill\break}
\newcommand{\CSLLeftMargin}[1]{\parbox[t]{\csllabelwidth}{#1}}
\newcommand{\CSLRightInline}[1]{\parbox[t]{\linewidth - \csllabelwidth}{#1}\break}
\newcommand{\CSLIndent}[1]{\hspace{\cslhangindent}#1}


\title{The Czech Constitutional Court database\thanks{Replication files
are available on the author's Github account
(\url{https://github.com/stepanpaulik/apex_courts_dataset/}).
\textbf{Current version}: December 09, 2023}}
\author{true}
\date{December 09, 2023}

% Jesus, okay, everything above this comment is default Pandoc LaTeX template. -----
% ----------------------------------------------------------------------------------
% I think I had assumed beamer and LaTex were somehow different templates.


\usepackage{kantlipsum}

\usepackage{abstract}
\renewcommand{\abstractname}{}    % clear the title
\renewcommand{\absnamepos}{empty} % originally center

\renewenvironment{abstract}
 {{%
    \setlength{\leftmargin}{0mm}
    \setlength{\rightmargin}{\leftmargin}%
  }%
  \relax}
 {\endlist}

\makeatletter
\def\@maketitle{%
  \newpage
%  \null
%  \vskip 2em%
%  \begin{center}%
  \let \footnote \thanks
      {\fontsize{18}{20}\selectfont\raggedright  \setlength{\parindent}{0pt} \@title \par}
    }
%\fi
\makeatother


\title{The Czech Constitutional Court database\thanks{Replication files
are available on the author's Github account
(\url{https://github.com/stepanpaulik/apex_courts_dataset/}).
\textbf{Current version}: December 09, 2023}  }

\date{}

\usepackage{titlesec}

% 
\titleformat*{\section}{\large\bfseries}
\titleformat*{\subsection}{\normalsize\itshape} % \small\uppercase
\titleformat*{\subsubsection}{\normalsize\itshape}
\titleformat*{\paragraph}{\normalsize\itshape}
\titleformat*{\subparagraph}{\normalsize\itshape}

% add some other packages ----------

% \usepackage{multicol}
% This should regulate where figures float
% See: https://tex.stackexchange.com/questions/2275/keeping-tables-figures-close-to-where-they-are-mentioned
\usepackage[section]{placeins}



\makeatletter
\@ifpackageloaded{hyperref}{}{%
\ifxetex
  \PassOptionsToPackage{hyphens}{url}\usepackage[setpagesize=false, % page size defined by xetex
              unicode=false, % unicode breaks when used with xetex
              xetex]{hyperref}
\else
  \PassOptionsToPackage{hyphens}{url}\usepackage[draft,unicode=true]{hyperref}
\fi
}

\@ifpackageloaded{color}{
    \PassOptionsToPackage{usenames,dvipsnames}{color}
}{%
    \usepackage[usenames,dvipsnames]{color}
}
\makeatother
\hypersetup{breaklinks=true,
            bookmarks=true,
            pdfauthor={Štěpán Paulík (Humboldt Universität zu Berlin,
Institut für Sozialwissenschaften,
\href{mailto:stepan.paulik.1@hu-berlin.de}{\nolinkurl{stepan.paulik.1@hu-berlin.de}})},
             pdfkeywords = {},
            pdftitle={The Czech Constitutional Court database},
            colorlinks=true,
            citecolor=blue,
            urlcolor=blue,
            linkcolor=magenta,
            pdfborder={0 0 0}}
\urlstyle{same}  % don't use monospace font for urls

% Add an option for endnotes. -----



% This will better treat References as a section when using natbib
% https://tex.stackexchange.com/questions/49962/bibliography-title-fontsize-problem-with-bibtex-and-the-natbib-package

% set default figure placement to htbp
\makeatletter
\def\fps@figure{htbp}
\makeatother



\usepackage{longtable}
\LTcapwidth=.95\textwidth
\linespread{1.05}
\usepackage{hyperref}

\newtheorem{hypothesis}{Hypothesis}


% trick for moving figures to back of document
% really wish we'd knock this shit off with moving tables/figures to back of document
% but, alas...

% 
% Optional code chunks ------
% SOURCE: https://stackoverflow.com/questions/50702942/does-rmarkdown-allow-captions-and-references-for-code-chunks



\begin{document}

% \textsf{\textbf{This is sans-serif bold text.}}
% \textbf{\textsf{This is bold sans-serif text.}}


% \maketitle

{% \usefont{T1}{pnc}{m}{n}
\setlength{\parindent}{0pt}
\thispagestyle{plain}
{%\fontsize{18}{20}\selectfont\raggedright
\maketitle  % title \par

}




{
   \vskip 13.5pt\relax \normalsize\fontsize{11}{12}
   \MakeUppercase{Štěpán Paulík}, \small{Humboldt Universität zu Berlin,
Institut für Sozialwissenschaften,
\href{mailto:stepan.paulik.1@hu-berlin.de}{\nolinkurl{stepan.paulik.1@hu-berlin.de}}}   

}

}






\vskip -8.5pt

{
\hypersetup{linkcolor=black}
\setcounter{tocdepth}{2}
\tableofcontents
}

 % removetitleabstract

{
\setcounter{tocdepth}{2}
\tableofcontents
}

\setlength{\parindent}{16pt}
\setlength{\parskip}{0pt}

% We'll put doublespacing here
% Remember to cut it out later before bib
\hypertarget{introduction}{%
\section{Introduction}\label{introduction}}

Although it has been traditionally espoused that there has been a divide
between the empirically oriented US legal scholarship, stemming from a
different perception of the role of courts and judges, and the rest of
the world (\protect\hyperlink{ref-hamannGermanFederalCourts2019}{Hamann
2019, 416}). Therein the judges have been viewed and empirically
researched as political
(\protect\hyperlink{ref-carrubbaWhoControlsContent2012}{Carrubba et al.
2012}; \protect\hyperlink{ref-clarkLocatingSupremeCourt2010}{Clark and
Lauderdale 2010};
\protect\hyperlink{ref-epsteinChoicesJusticesMake1997}{Epstein and
Knight 1997};
\protect\hyperlink{ref-lauderdaleScalingPoliticallyMeaningful2014}{Lauderdale
and Clark 2014};
\protect\hyperlink{ref-sunsteinAreJudgesPolitical2006}{Sunstein et al.
2006}) or strategic
(\protect\hyperlink{ref-cameronChapterWhatJudges2017}{Cameron and
Kornhauser 2017};
\protect\hyperlink{ref-clarkEstimatingEffectLeisure2018}{Clark, Engst,
and Staton 2018};
\protect\hyperlink{ref-epsteinWhyWhenJudges2011}{Epstein, Landes, and
Posner 2011};
\protect\hyperlink{ref-epsteinStrategicRevolutionJudicial2000}{Epstein
and Knight 2000};
\protect\hyperlink{ref-kornhauserModelingCollegialCourts1992}{Kornhauser
1992b},
\protect\hyperlink{ref-kornhauserModelingCollegialCourts1992a}{1992a};
\protect\hyperlink{ref-posnerWhatJudgesJustices1993}{Posner 1993},
\protect\hyperlink{ref-posnerHowJudgesThink2010}{2010};
\protect\hyperlink{ref-rousseyOverburdenedJudges2018}{Roussey and
Soubeyran 2018}) actors.

In contrast to, especially in European legal systems, such as the one at
hand - Czechia, judges have been perceived as ``proclaimers of law'' and
the law handed down by them
(\protect\hyperlink{ref-hamannGermanFederalCourts2019}{Hamann 2019,
417}). Such a view had hindered empirical legal research in Europe. The
lack of empirical legal research could also be partially blamed on lack
of high quality data, a prerequisite for any quantitative empirical
research. At least so the story goes until recently. The interest in
empirical legal studies has picked up in the last years across the whole
continent, including studies on plethora of topics within Germany
(\protect\hyperlink{ref-arnoldScalingCourtDecisions2023}{Arnold, Engst,
and Gschwend 2023};
\protect\hyperlink{ref-coupetteQuantitativeRechtswissenschaft2018}{Coupette
and Fleckner 2018};
\protect\hyperlink{ref-engstEinflussParteinaheAuf2017}{Benjamin G. Engst
et al. 2017};
\protect\hyperlink{ref-wittigOccurrenceSeparateOpinions2016}{Wittig
2016}), Spain and Portugal
(\protect\hyperlink{ref-hanrettyDissentIberiaIdeal2012}{Hanretty 2012}),
the UK
(\protect\hyperlink{ref-hanrettyCourtSpecialistsJudicial2020}{Hanretty
2020}) or the EU institutions
(\protect\hyperlink{ref-bielenBacklogsLitigationRates2018}{Bielen et al.
2018}; \protect\hyperlink{ref-brekkeThatOrderHow2023}{Brekke, Naurin, et
al. 2023}; \protect\hyperlink{ref-fjelstulHowChamberSystem2023}{J.
Fjelstul 2023};
\protect\hyperlink{ref-fjelstulEvolutionEuropeanUnion2019}{Joshua C.
Fjelstul 2019};
\protect\hyperlink{ref-fjelstulTimelyAdministrationJustice2022}{Joshua
C. Fjelstul, Gabel, and Carrubba 2022}).

Publications of new high quality publicly accessible data have gone hand
in hand with these developments. To the many released comprehensive data
sets in the recent years belong namely the Iuropa project's CJEU
database (\protect\hyperlink{ref-brekkeCJEUDatabasePlatform2023}{Brekke,
Fjelstul, et al. 2023}), the German Federal courts
(\protect\hyperlink{ref-hamannGermanFederalCourts2019}{Hamann 2019}) as
well as the Constitutional Court
(\protect\hyperlink{ref-engstConstitutionalCourtDatabaseForthcoming}{Benjamin
G. Engst, Hönnige, and Gschwend Forthcoming})

To the author's best knowledge, there have been solitary attempts go
gather data in some shape or form in the Czech context
(\protect\hyperlink{ref-harastaAnnotatedCorpusCzech2018}{Harašta et al.
2018}; \protect\hyperlink{ref-novotnaCzechCourtDecisions2019}{Novotná
and Harašta 2019}), mainly thanks to the Institute of Law and Technology
based in Brno, as well as isolated attempts to conduct quantitative
methods or research employing natural language processing and alike
methods (\protect\hyperlink{ref-chmelZpravodajoveSenatyVliv2017}{Chmel
2017};
\protect\hyperlink{ref-eliasekAutomatickaKlasifikaceVyznamovych2020}{Eliášek,
Kól, and Švaňa 2020};
\protect\hyperlink{ref-harastaCitacniAnalyzaJudikatury2021}{Harašta et
al. 2021};
\protect\hyperlink{ref-vartazaryanSitOvaAnalyza2022}{Vartazaryan 2022}).
Unfortunately, the former group does not always adhere to the principles
of of high-quality infrastructure, namely the principle of
foundationality, espoused by Weinshall and Epstein
(\protect\hyperlink{ref-weinshallDevelopingHighQualityData2020}{2020}),
p.~424, the latter group did not publish data/code at all. The output of
the law and economics/criminology team centered around Jakub Drápal and
Libor Dušek stands out as the only systemic effort to conduct replicable
quantitative legal research in Czechia
(\protect\hyperlink{ref-drapalSentencingCzechRepublic2021}{Drápal 2021},
\protect\hyperlink{ref-drapalPunitiveNegligenceMyths2023}{2023};
\protect\hyperlink{ref-drapalUsingLargeLanguage2023}{Drápal, Westermann,
and Savelka 2023};
\protect\hyperlink{ref-drapalLawAuthorityHow2023}{Drápal and Dusek
2023}; \protect\hyperlink{ref-drapalWhatValueJudicial2023}{Drápal and
Pina-Sánchez 2023};
\protect\hyperlink{ref-drapalSentencingDecisionsQuantity2023}{Drápal and
Šoltés 2023}). The data they are working with currently are still
unpublished.

Therefore, the effort to put together and to publish a high-quality
dataset on the CCC is more than warranted. The article proceeds as
follow. In the section X, I introduce the CCC, namely its compositions,
its internal organization and its powers. In the section X, I introduce
the CCC dataset. Therein, I briefly discuss its structure, describe the
variables. The section X then discusses the adherence of the CCC dataset
to five principles of a high-quality dataset, including its relevance
for research, as well as to the adherence of the tidy data principles.
The last section concludes.

\hypertarget{a-brief-introduction-of-the-ccc}{%
\section{A brief introduction of the
CCC}\label{a-brief-introduction-of-the-ccc}}

The CCC consists of fifteen justices, out of which one is the president
of the CCC, two are vice presidents and twelve associate justices
(following the terminology of
\protect\hyperlink{ref-kosarConstitutionalCourtCzechia2020}{Kosař and
Vyhnánek 2020}). These fifteen justices are appointed by the president
of the Czech republic upon approval of the Senate. The justices enjoy 10
years terms with the possibility of re-election. The three CCC
functionaries are unilaterally appointed by the Czech president.

Regarding the competences, the CCC is a typical Kelsenian court inspired
mainly by the German Federal Constitutional Court. The CCC enjoys the
power of abstract constitutional review, including constitutional
amendments. The abstract review procedure is initiated by political
actors (for example MPs) and usually concerns political issues.
Moreover, an ordinary court can initiate a concrete review procedure, if
that court reaches the conclusion that a legal norm upon which its
decision depends is not compatible with the constitution. Individuals
can also lodge constitutional complaints before the CCC. Lastly, the CCC
can also resolve separation-of-powers disputes, it can \emph{ex ante}
review international treaties, decide on impeachment of the president of
the republic, and it has additional ancillary powers (for a complete
overview, see
\protect\hyperlink{ref-kosarConstitutionalCourtCzechia2020}{Kosař and
Vyhnánek 2020}).

The CCC is an example of a collegial court. From the perspective of the
inner, the CCC can decide in four bodies: (1) individual justices, (2)
3-member chambers (\emph{senáty}), (3) the plenum (\emph{plénum}), and
(4) special disciplinary chambers. However, the 3-member chambers and
the plenum play a crucial role. The plenum is composed of all justices,
whereas the four 3-member chambers are composed of the associate
justices. Neither the president of the CCC or her vice-presidents are
permanents members of the 3-member chambers. Until 2016, the composition
of the chambers was static. However, in 2016, a system of regular
2-yearly rotations was introduced, wherein the president of the chamber
rotates to a different every 2 years. I am of the view that such a
institutional change opens up potential for quasi-experimental research
similar to the Gschwend, Sternberg, and Zittlau
(\protect\hyperlink{ref-gschwendAreJudgesPolitical2016}{2016}) study
utilizing judge absences within the 3-member panels of the German
federal constitutional court.

In the chamber proceedings, decisions on admissibility must be
unanimous, decisions on merits need not be, therefore, two votes are
necessary.\footnote{Which allows enables separate opinions} In the
plenum, the general voting quorum is a simple majority and the plenum is
quorate when there are ten justices present. The abstract review is one
of the exceptions that sets the quorum higher, at 9 votes more
specifically.

A judge rapporteur plays a crucial role. Each case of the CCC gets
assigned to a judge raporteur. The assignement is regulated by a case
allocation plan (the original term is \emph{rozvrh práce}, which is
usually translated as a \emph{work schedule}, however, I borrow the term
\emph{case allocation plan} from Hamann
(\protect\hyperlink{ref-hamannGermanFederalCourts2019}{2019}), p.~673)
She is tasked with drafting the opinion, about which the body then
votes. The president of the CCC (in plenary cases) or the president of
the chamber (in chamber cases) may re-assign a case to a different judge
rapporteur if the draft opinion by the original judge rapporteur did not
receive a majority of votes. Unfortunately, the CCC does not keep track
of these reassignments.\footnote{I unsuccessfully attempted to retrieve
  the information with the right to information}

The CCC allows for separate opinions. They can take two forms:
dissenting or concurring opinions. Each justice has the right to author
a separate opinion, which then gets published with the CCC decision. It
follows that not every anti-majority vote implies a separate opinion, it
is up to the justices to decide whether they want to attach a separate
opinion with their vote.

It may be concluded that the CCC takes after the american model of
selection of justices, with the president of the republic and the upper
chamber being in the spotlight, but is also a typical example of a
Kelsenian specialised court with concentrated constitutional review. The
CCC stands out in how powerful its constitutional review is, having
attracted the power to review even constitutional amendments, thus, the
CCC is a powerful player in the Czech political system. The internal
organization of the CCC gives a lot of room for strategic considerations
of its justices. Not only due to the similarities with the
constitutional adjudication powerhouses but also due to its own
idiosyncrasies, I believe the CCC to be a worthy object of empirical
legal research.

\hypertarget{principles-guiding-the-creation-of-the-dataset}{%
\section{Principles guiding the creation of the
dataset}\label{principles-guiding-the-creation-of-the-dataset}}

The Czech Constitutional Court database is a ``multi-user dataset''
created in a principled manner. Epstein et al.
(\protect\hyperlink{ref-epsteinIntroductionEmpiricalLegal2014}{2014}),
p.~14 define a multi-user dataset as a dataset created with the purpose
of that ``{[}r{]}ather than collect data to answer particular research
questions {[}\ldots{]} the idea is to amass a dataset so rich in content
that mzu ultiple users, even those with distinct projects, can draw on
it.''

Accordingly, the Czech Constitutional Court dataset upholds the
principles of a high-quality dataset espoused by Weinshall and Epstein
(\protect\hyperlink{ref-weinshallDevelopingHighQualityData2020}{2020}),
p.~424, namely that the database is (1) capable of addressing real-world
problems, (2) accessible, (3) reproducible and reliable, (4) sustainable
and updatable, and (5) foundational. The data structure also follows the
principles of tidy data. According to Wickham
(\protect\hyperlink{ref-wickhamTidyData2014}{2014}), tidy data are data
with such a tabular\footnote{i.e.~with a column and row structure}
structure that

\begin{enumerate}
\def\labelenumi{(\arabic{enumi})}
\tightlist
\item
  every column is a variable,
\item
  every row is an observation,
\item
  every cell is a single value.
\end{enumerate}

Although the dataset contains one all-encompassing ``master'' table,
some of the variables (for example concerned laws or applicant) contain
multiple values nested in one cell. The reason is simple: the master
table contains observations on the case-level, whereas whenever a
variable contains more values per case, the data structure would then
require a variable-case level. For this purpose, the dataset is also
split up into multiple smaller tables on a variable-case level (for
example dissenting judge-case level or concerned acts-case level), which
can then be joined together relational database SQL-style in the form of
unique keys.

\hypertarget{capacity-to-address-real-world-problems}{%
\subsection{Capacity to Address Real-World
Problems}\label{capacity-to-address-real-world-problems}}

As Kosař and Vyhnánek
(\protect\hyperlink{ref-kosarConstitutionalCourtCzechia2020}{2020})
argue, the clerks at the CCC play an especially vital and unappreciated
role: ``The initial idea of the legislature was to grant each Justice
one law clerk who would take administrative burdens unrelated to
substantive decision-making off the Justices' shoulders. Yet the reality
is different. First, due to the growing caseload, the number of law
clerks per Justice increased gradually; today, each Justice has three
law clerks. Moreover, law clerks de facto prepare drafts of most CCC
judgments and decisions, and the real administrative burden has been
`outsourced' to secretaries of the cabinets.'' The difficulty of
studying the role of clerks was highlighted in the Clark, Engst, and
Staton (\protect\hyperlink{ref-clarkEstimatingEffectLeisure2018}{2018})
study on the effects of leisure on judicial performance.

The CCC dataset contains the information on all clerks that have served
on the CCC, including their gender, education, beginning and end of
mandates. Such an information can serve as a basis for any research on
the role of clerks. For the purpose of showing the capability of solving
real-world problems, I present descriptive statistics.

\hypertarget{accessibility}{%
\subsection{Accessibility}\label{accessibility}}

\hypertarget{reliability-and-reproducibility}{%
\subsection{Reliability and
Reproducibility}\label{reliability-and-reproducibility}}

\hypertarget{sustainability}{%
\subsection{Sustainability}\label{sustainability}}

\hypertarget{foundational}{%
\subsection{Foundational}\label{foundational}}

\vspace{30pt}

\hypertarget{literature}{%
\section*{Literature}\label{literature}}
\addcontentsline{toc}{section}{Literature}

\hypertarget{refs}{}
\begin{CSLReferences}{1}{0}
\leavevmode\vadjust pre{\hypertarget{ref-arnoldScalingCourtDecisions2023}{}}%
Arnold, Christian, Benjamin G. Engst, and Thomas Gschwend. 2023.
{``Scaling {Court Decisions} with {Citation Networks}.''} \emph{Journal
of Law and Courts} 11 (1): 25--44. \url{https://doi.org/10.1086/717420}.

\leavevmode\vadjust pre{\hypertarget{ref-bielenBacklogsLitigationRates2018}{}}%
Bielen, Samantha, Ludo Peeters, Wim Marneffe, and Lode Vereeck. 2018.
{``Backlogs and Litigation Rates: {Testing} Congestion Equilibrium
Across {European} Judiciaries.''} \emph{International Review of Law and
Economics} 53 (March): 9--22.
\url{https://doi.org/10.1016/j.irle.2017.09.002}.

\leavevmode\vadjust pre{\hypertarget{ref-brekkeCJEUDatabasePlatform2023}{}}%
Brekke, Stein Arne, Joshua C. Fjelstul, Silje Synnøve Lyder Hermansen,
and Daniel Naurin. 2023. {``The {CJEU Database Platform}: {Decisions}
and {Decision-Makers}.''} \emph{Journal of Law and Courts}, January,
1--22. \url{https://doi.org/10.1017/jlc.2022.3}.

\leavevmode\vadjust pre{\hypertarget{ref-brekkeThatOrderHow2023}{}}%
Brekke, Stein Arne, Daniel Naurin, Urška Šadl, and Lucía López-Zurita.
2023. {``That's an {Order}! {How} the {Quest} for {Efficiency Is
Transforming Judicial Cooperation} in {Europe}.''} \emph{JCMS: Journal
of Common Market Studies} 61 (1): 58--75.
\url{https://doi.org/10.1111/jcms.13346}.

\leavevmode\vadjust pre{\hypertarget{ref-cameronChapterWhatJudges2017}{}}%
Cameron, Charles M., and Lewis A. Kornhauser. 2017. {``Chapter 3: {What
Do Judges Want}? {How} to {Model Judicial Preferences}.''} SSRN
Scholarly Paper. {Rochester, NY}. June 2, 2017.
\url{https://doi.org/10.2139/ssrn.2979419}.

\leavevmode\vadjust pre{\hypertarget{ref-carrubbaWhoControlsContent2012}{}}%
Carrubba, Cliff, Barry Friedman, Andrew D. Martin, and Georg Vanberg.
2012. {``Who {Controls} the {Content} of {Supreme Court Opinions}?''}
\emph{American Journal of Political Science} 56 (2): 400--412.
\url{https://doi.org/10.1111/j.1540-5907.2011.00557.x}.

\leavevmode\vadjust pre{\hypertarget{ref-chmelZpravodajoveSenatyVliv2017}{}}%
Chmel, Jan. 2017. {``Zpravodajové a Senáty: {Vliv} Složení Senátu Na
Rozhodování {Ústavního} Soudu {České} Republiky o Ústavních
Stížnostech.''} \emph{Časopis Pro Právní Vědu a Praxi} 25 (4): 739.
\url{https://doi.org/10.5817/CPVP2017-4-9}.

\leavevmode\vadjust pre{\hypertarget{ref-clarkEstimatingEffectLeisure2018}{}}%
Clark, Tom S., Benjamin G. Engst, and Jeffrey K. Staton. 2018.
{``Estimating the {Effect} of {Leisure} on {Judicial Performance}.''}
\emph{The Journal of Legal Studies} 47 (2): 349--90.
\url{https://doi.org/10.1086/699150}.

\leavevmode\vadjust pre{\hypertarget{ref-clarkLocatingSupremeCourt2010}{}}%
Clark, Tom S., and Benjamin Lauderdale. 2010. {``Locating {Supreme Court
Opinions} in {Doctrine Space}.''} \emph{American Journal of Political
Science} 54 (4): 871--90.
\url{https://doi.org/10.1111/j.1540-5907.2010.00470.x}.

\leavevmode\vadjust pre{\hypertarget{ref-coupetteQuantitativeRechtswissenschaft2018}{}}%
Coupette, Corinna, and Andreas M. Fleckner. 2018. {``Quantitative
{Rechtswissenschaft}.''} \emph{JuristenZeitung (JZ)} 73 (8): 379--89.
\url{https://doi.org/10.1628/jz-2018-0020}.

\leavevmode\vadjust pre{\hypertarget{ref-drapalSentencingCzechRepublic2021}{}}%
Drápal, Jakub. 2021. {``Sentencing in the {Czech Republic}: {An
Empirical Investigation},''} March.
\url{https://dspace.cuni.cz/handle/20.500.11956/148293}.

\leavevmode\vadjust pre{\hypertarget{ref-drapalPunitiveNegligenceMyths2023}{}}%
---------. 2023. {``Punitive by Negligence? {The} Myths and Reality of
Penal Nationalism in the {Czech Republic}.''} \emph{European Journal of
Criminology} 20 (4): 1549--67.
\url{https://doi.org/10.1177/14773708211063753}.

\leavevmode\vadjust pre{\hypertarget{ref-drapalLawAuthorityHow2023}{}}%
Drápal, Jakub, and Libor Dusek. 2023. {``Law or {Authority}: {How Penal
Elites Shape Sentencing Policy} by {Non-Binding Interventions}.''} SSRN
Scholarly Paper. {Rochester, NY}. October 31, 2023.
\url{https://doi.org/10.2139/ssrn.4619030}.

\leavevmode\vadjust pre{\hypertarget{ref-drapalWhatValueJudicial2023}{}}%
Drápal, Jakub, and Jose Pina-Sánchez. 2023. {``What Is the {Value} of
{Judicial Experience}? {Exploring Judge Trajectories Using Longitudinal
Data}.''} \emph{Justice Quarterly} 40 (2): 211--40.
\url{https://doi.org/10.1080/07418825.2022.2051585}.

\leavevmode\vadjust pre{\hypertarget{ref-drapalSentencingDecisionsQuantity2023}{}}%
Drápal, Jakub, and Michal Šoltés. 2023. {``Sentencing Decisions Around
Quantity Thresholds: Theory and Experiment.''} \emph{Journal of
Experimental Criminology}, July.
\url{https://doi.org/10.1007/s11292-023-09568-8}.

\leavevmode\vadjust pre{\hypertarget{ref-drapalUsingLargeLanguage2023}{}}%
Drápal, Jakub, Hannes Westermann, and Jaromir Savelka. 2023. {``Using
{Large Language Models} to {Support Thematic Analysis} in {Empirical
Legal Studies}.''} October 28, 2023.
\url{https://doi.org/10.48550/arXiv.2310.18729}.

\leavevmode\vadjust pre{\hypertarget{ref-eliasekAutomatickaKlasifikaceVyznamovych2020}{}}%
Eliášek, Martin, Jakub Kól, and Miloš Švaňa. 2020. {``Automatická
Klasifikace Významových Celků v Judikatuře.''} \emph{Revue Pro Právo a
Technologie} 11 (21): 3--20. \url{https://doi.org/10.5817/RPT2020-1-1}.

\leavevmode\vadjust pre{\hypertarget{ref-engstEinflussParteinaheAuf2017}{}}%
Engst, Benjamin G., Thomas Gschwend, Nils Schaks, Sebastian Sternberg,
and Caroline Wittig. 2017. {``Zum {Einfluss} Der {Parteinähe} Auf Das
{Abstimmungsverhalten} Der {Bundesverfassungsrichter} -- Eine
Quantitative {Untersuchung}.''} \emph{JuristenZeitung} 72 (17): 816--26.
\url{https://www.jstor.org/stable/44867374}.

\leavevmode\vadjust pre{\hypertarget{ref-engstConstitutionalCourtDatabaseForthcoming}{}}%
Engst, Benjamin G, Christoph Hönnige, and Thomas Gschwend. Forthcoming.
{``The {Constitutional Court Database}.''} \emph{Working Paper},
Forthcoming, 39.

\leavevmode\vadjust pre{\hypertarget{ref-epsteinChoicesJusticesMake1997}{}}%
Epstein, Lee, and Jack Knight. 1997. \emph{The {Choices Justices Make}}.
{SAGE}. \url{https://books.google.com?id=hSnom2h2_zUC}.

\leavevmode\vadjust pre{\hypertarget{ref-epsteinStrategicRevolutionJudicial2000}{}}%
---------. 2000. {``Toward a {Strategic Revolution} in {Judicial
Politics}: {A Look Back}, {A Look Ahead}.''} \emph{Political Research
Quarterly} 53 (3): 625--61.
\url{https://doi.org/10.1177/106591290005300309}.

\leavevmode\vadjust pre{\hypertarget{ref-epsteinWhyWhenJudges2011}{}}%
Epstein, Lee, William M. Landes, and Richard A. Posner. 2011. {``Why
({And When}) {Judges Dissent}: {A Theoretical And Empirical
Analysis}.''} \emph{Journal of Legal Analysis} 3 (1): 101--37.
\url{https://doi.org/10.1093/jla/3.1.101}.

\leavevmode\vadjust pre{\hypertarget{ref-epsteinIntroductionEmpiricalLegal2014}{}}%
Epstein, Lee, Andrew D. Martin, Lee Epstein, and Andrew D. Martin. 2014.
\emph{An {Introduction} to {Empirical Legal Research}}. {Oxford, New
York}: {Oxford University Press}.

\leavevmode\vadjust pre{\hypertarget{ref-fjelstulHowChamberSystem2023}{}}%
Fjelstul, Joshua. 2023. {``How the {Chamber System} at the {CJEU
Undermines} the {Consistency} of the {Court}'s {Application} of {EU
Law}.''} \emph{Journal of Law and Courts}, 717422.
\url{https://doi.org/10.1086/717422}.

\leavevmode\vadjust pre{\hypertarget{ref-fjelstulEvolutionEuropeanUnion2019}{}}%
Fjelstul, Joshua C. 2019. {``The Evolution of {European Union} Law: {A}
New Data Set on the {\emph{Acquis Communautaire}}.''} \emph{European
Union Politics} 20 (4): 670--91.
\url{https://doi.org/10.1177/1465116519842947}.

\leavevmode\vadjust pre{\hypertarget{ref-fjelstulTimelyAdministrationJustice2022}{}}%
Fjelstul, Joshua C., Matthew Gabel, and Clifford J. Carrubba. 2022.
{``The Timely Administration of Justice: Using Computational Simulations
to Evaluate Institutional Reforms at the {CJEU}.''} \emph{Journal of
European Public Policy}, August, 1--22.
\url{https://doi.org/10.1080/13501763.2022.2113115}.

\leavevmode\vadjust pre{\hypertarget{ref-gschwendAreJudgesPolitical2016}{}}%
Gschwend, Thomas, Sebastian Sternberg, and Steffen Zittlau. 2016. {``Are
{Judges Political Animals} After {All}? {Quasi-Experimental Evidence}
from the {German Federal Constitutional Court}.''} SSRN Scholarly Paper.
{Rochester, NY}. February 26, 2016.
\url{https://doi.org/10.2139/ssrn.2738512}.

\leavevmode\vadjust pre{\hypertarget{ref-hamannGermanFederalCourts2019}{}}%
Hamann, Hanjo. 2019. {``The {German Federal Courts Dataset} 1950--2019:
{From Paper Archives} to {Linked Open Data}.''} \emph{Journal of
Empirical Legal Studies} 16 (3): 671--88.
\url{https://doi.org/10.1111/jels.12230}.

\leavevmode\vadjust pre{\hypertarget{ref-hanrettyDissentIberiaIdeal2012}{}}%
Hanretty, Chris. 2012. {``Dissent in {Iberia}: {The} Ideal Points of
Justices on the {Spanish} and {Portuguese Constitutional Tribunals}.''}
\emph{European Journal of Political Research} 51 (5): 671--92.
\url{https://doi.org/10.1111/j.1475-6765.2012.02056.x}.

\leavevmode\vadjust pre{\hypertarget{ref-hanrettyCourtSpecialistsJudicial2020}{}}%
---------. 2020. \emph{A {Court} of {Specialists}: {Judicial Behavior}
on the {UK Supreme Court}}. {Oxford University Press}.
\url{https://doi.org/10.1093/oso/9780197509234.001.0001}.

\leavevmode\vadjust pre{\hypertarget{ref-harastaAnnotatedCorpusCzech2018}{}}%
Harašta, Jakub, Jaromír Šavelka, František Kasl, Adéla Kotková, Pavel
Loutocký, Jakub Míšek, Daniela Procházková, et al. 2018. {``Annotated
{Corpus} of {Czech Case Law} for {Reference Recognition Tasks}.''} In
\emph{Text, {Speech}, and {Dialogue}}, edited by Petr Sojka, Aleš Horák,
Ivan Kopeček, and Karel Pala, 11107:239--50. {Cham}: {Springer
International Publishing}.
\url{https://doi.org/10.1007/978-3-030-00794-2_26}.

\leavevmode\vadjust pre{\hypertarget{ref-harastaCitacniAnalyzaJudikatury2021}{}}%
Harašta, Jakub, Terezie Smejkalová, Jaromír Šavelka, and Radim Polčák.
2021. \emph{Citační analýza judikatury}. Vydání první. Právní
monografie. {Praha}: {Wolters Kluwer}.

\leavevmode\vadjust pre{\hypertarget{ref-kornhauserModelingCollegialCourts1992a}{}}%
Kornhauser, Lewis A. 1992a. {``Modeling {Collegial Courts}. {II}. {Legal
Doctrine}.''} \emph{Journal of Law, Economics and Organization} 8: 441.
\url{https://heinonline.org/HOL/Page?handle=hein.journals/jleo8&id=449&div=&collection=}.

\leavevmode\vadjust pre{\hypertarget{ref-kornhauserModelingCollegialCourts1992}{}}%
---------. 1992b. {``Modeling Collegial Courts {I}:
{Path-dependence}.''} \emph{International Review of Law and Economics}
12 (2): 169--85. \url{https://doi.org/10.1016/0144-8188(92)90034-O}.

\leavevmode\vadjust pre{\hypertarget{ref-kosarConstitutionalCourtCzechia2020}{}}%
Kosař, David, and Ladislav Vyhnánek. 2020. {``The {Constitutional Court}
of {Czechia}.''} In \emph{The {Max Planck Handbooks} in {European Public
Law}: {Volume III}: {Constitutional Adjudication}: {Institutions}},
edited by Armin von Bogdandy, Peter Huber, and Christoph Grabenwarter,
0. {Oxford University Press}.
\url{https://doi.org/10.1093/oso/9780198726418.003.0004}.

\leavevmode\vadjust pre{\hypertarget{ref-lauderdaleScalingPoliticallyMeaningful2014}{}}%
Lauderdale, Benjamin E., and Tom S. Clark. 2014. {``Scaling {Politically
Meaningful Dimensions Using Texts} and {Votes}: {SCALING POLITICALLY
MEANINGFUL DIMENSIONS}.''} \emph{American Journal of Political Science}
58 (3): 754--71. \url{https://doi.org/10.1111/ajps.12085}.

\leavevmode\vadjust pre{\hypertarget{ref-novotnaCzechCourtDecisions2019}{}}%
Novotná, Tereza, and Jakub Harašta. 2019. {``The {Czech Court Decisions
Corpus} ({CzCDC}): {Availability} as the {First Step}.''} October 21,
2019. \url{http://arxiv.org/abs/1910.09513}.

\leavevmode\vadjust pre{\hypertarget{ref-posnerWhatJudgesJustices1993}{}}%
Posner, Richard A. 1993. \emph{What {Do Judges} and {Justices
Maximize}?: (The {Same Thing Everyone Else Does})}. {Law School,
University of Chicago}. \url{https://books.google.com?id=ciFUHQAACAAJ}.

\leavevmode\vadjust pre{\hypertarget{ref-posnerHowJudgesThink2010}{}}%
---------. 2010. \emph{How {Judges Think}}. {Harvard University Press}.
\url{https://books.google.com?id=ZVUC8riEVPQC}.

\leavevmode\vadjust pre{\hypertarget{ref-rousseyOverburdenedJudges2018}{}}%
Roussey, Ludivine, and Raphael Soubeyran. 2018. {``Overburdened
Judges.''} \emph{International Review of Law and Economics} 55
(September): 21--32. \url{https://doi.org/10.1016/j.irle.2018.02.003}.

\leavevmode\vadjust pre{\hypertarget{ref-sunsteinAreJudgesPolitical2006}{}}%
Sunstein, Cass R., David Schkade, Lisa M. Ellman, and Andres Sawicki.
2006. \emph{Are {Judges Political}? {An Empirical Analysis} of the
{Federal Judiciary}}. {Brookings Institution Press}.
\url{https://www.jstor.org/stable/10.7864/j.ctt12879t7}.

\leavevmode\vadjust pre{\hypertarget{ref-vartazaryanSitOvaAnalyza2022}{}}%
Vartazaryan, Gor. 2022. {``Sít'ová Analỳza Disentujících Ústavních
Soudců.''} \emph{Pravnik}, no. 12.

\leavevmode\vadjust pre{\hypertarget{ref-weinshallDevelopingHighQualityData2020}{}}%
Weinshall, Keren, and Lee Epstein. 2020. {``Developing {High-Quality
Data Infrastructure} for {Legal Analytics}: {Introducing} the {Israeli
Supreme Court Database}.''} \emph{Journal of Empirical Legal Studies} 17
(2): 416--34. \url{https://doi.org/10.1111/jels.12250}.

\leavevmode\vadjust pre{\hypertarget{ref-wickhamTidyData2014}{}}%
Wickham, Hadley. 2014. {``Tidy {Data}.''} \emph{Journal of Statistical
Software} 59 (10). \url{https://doi.org/10.18637/jss.v059.i10}.

\leavevmode\vadjust pre{\hypertarget{ref-wittigOccurrenceSeparateOpinions2016}{}}%
Wittig, Caroline. 2016. \emph{The {Occurrence} of {Separate Opinions} at
the {Federal Constitutional Court}}. {Logos Verlag Berlin}.
\url{https://doi.org/10.30819/4411}.

\end{CSLReferences}

\end{document}
