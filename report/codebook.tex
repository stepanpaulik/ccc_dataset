%--------------------------------------------------%
% generated by the codebookr R package
% created by Joshua C. Fjelstul, Ph.D.
%--------------------------------------------------%

\documentclass[10pt]{article}

%--------------------------------------------------%
% packages
%--------------------------------------------------%

% page layout
\usepackage{geometry}

% fonts
\usepackage[english]{babel}
\usepackage{underscore}
\usepackage{anyfontsize}
\usepackage[utf8]{inputenc}
\usepackage[T1]{fontenc}
\usepackage{fontspec}

% graphics and tables
\usepackage{graphicx} % add figures
\usepackage{xcolor} % change font color
\usepackage{tikz} % add graphics

% paragraph spacing
\usepackage{setspace}

% hyperlinks
\usepackage{url}

% table of contents
\usepackage{tocloft}

% test alignment
\usepackage{ragged2e}

% multi-page tables
\usepackage{longtable}

% custom lists
\usepackage{enumitem}

% insert content on every page
\usepackage{atbegshi} 

% code formatting
\usepackage{tcolorbox}

%--------------------------------------------------%
% colors
%--------------------------------------------------%

% define colors
\definecolor{themecolor}{HTML}{FFF0F7}
\definecolor{background}{HTML}{EEF6FD}

% format hyperlinks
\usepackage[colorlinks=true,linkcolor=themecolor,citecolor=themecolor,urlcolor=themecolor,breaklinks=true]{hyperref}

%--------------------------------------------------%
% formatting
%--------------------------------------------------%

% configure main font
\setmainfont[Ligatures=TeX,BoldFont={Roboto Medium}]{Roboto Light}
\setmonofont[Ligatures=TeX]{Roboto Mono-Light}

% set page margins
\geometry{top = 1.5in, bottom = 1.5in, left = 1.5in, right = 1.5in}

% set paper size
\geometry{letterpaper}

% format table of contents
\renewcommand{\cftsecdotsep}{10}
\renewcommand{\cftsecleader}{\cftdotfill{\cftdotsep}}
\renewcommand{\cftsecfont}{{\small\color{black!75}\bfseries}}
\renewcommand{\cftsecpagefont}{{\small\color{black!75}\normalfont}}

% adjust spacing
\usepackage{parskip}
\parskip=10pt
\renewcommand{\baselinestretch}{1.4}

% hyphen formatting
\hyphenpenalty = 10000
\exhyphenpenalty = 10000

% prevent widow and orphan lines
\widowpenalty10000
\clubpenalty10000

%--------------------------------------------------%
% page elements
%--------------------------------------------------%

% a command to make a code box
\newtcbox{\codebox}{nobeforeafter,tcbox raise base,colback=black!5,colframe=white,coltext=black!75,boxrule=0pt,arc=3pt,boxsep=0pt,
left=4pt,right=4pt,top=3pt,bottom=3pt}

% a command to make a chip
\newtcbox{\chip}{nobeforeafter,tcbox raise base,colback=black!5,colframe=white,coltext=black!75,boxrule=0pt,arc=11pt,boxsep=0pt,
left=10pt,right=10pt,top=8pt,bottom=8pt}

% command to format code
\newcommand{\code}[1]{\codebox{{\footnotesize\texttt{#1}}}}

% command to highlight text
\newcommand{\highlight}[1]{{\color{themecolor} \textbf{#1}}}

% command to create a divider
\newcommand{\dividerline}{{\color{gray!10} \rule[4pt] {\textwidth}{3pt}}}

% command to add a cover
\newcommand{\cover}[4]{
\begin{tikzpicture}[remember picture,overlay, shift={(current page.south west)}]
\fill[themecolor] (0, 5.5in) rectangle ++ (8.5in, 5.5in); % header bar
\fill[black!5] (0, 4in) rectangle ++ (8.5in, 1.5in); % middle bar
\fill[white] (0, 0in) rectangle ++ (8.5in, 4in); % footer bar
\node[anchor=west] at (1.5in, 6.25in) {\color{white} \fontsize{60}{60}\selectfont \begin{minipage}{5.5in} \textbf{Codebook} \fontsize{15}{15}\selectfont \hspace{5pt} v #2 \end{minipage}};
\node[anchor=west, align=left] at (1.5in, 4.75in) {\begin{minipage}{5.5in} \color{black!40} \fontsize{#4}{#4} \selectfont #1 \end{minipage}};
\node[anchor=west, align=left, minimum height=2in] at (1.5in, 2.55in) {\begin{minipage}[t][2in]{5.5in} \color{black!40} \fontsize{10}{10} \selectfont #3 \end{minipage}};
\end{tikzpicture}
}

% command to add a header page
\newcommand{\headerpage}[4]{
	\newpage
	\begin{tikzpicture}[remember picture,overlay, shift={(current page.south west)}]
		\fill[themecolor] (0, 9in) rectangle ++ (8.5in, 2in); % header line 1
		\fill[black!5] (0, 8in) rectangle ++ (8.5in, 1in); % header line 2
		\node[anchor = west] at (1.5in, 9.6in) {\color{white} \fontsize{#3}{#3}\selectfont \textbf{#1}}; % heading
		\node[anchor = west] at (1.5in, 8.5in) {\color{black!40} \fontsize{#4}{#4}\selectfont #2}; % heading
	\end{tikzpicture}
	\phantomsection
	\addcontentsline{toc}{section}{#1}
	\vspace{1.5in}
}

% command to layout page
\newcommand\pagelayout{
	\begin{tikzpicture}[remember picture,overlay, shift={(current page.south west)}]
		% \fill[themecolor] (0, 10.75in) rectangle ++ (8.5in, 0.25in); % header
		\fill[black!5] (0, 0) rectangle ++ (8.5in, 0.5in); % footer
		\draw (0.25in, 0.25in) node[anchor = west] {\fontsize{9}{9}\selectfont \color{black!40} The Czech Constitutional Court Dataset \hspace{5pt} | \hspace{5pt} Štěpán Paulík}; % footer content
		\draw (8.25in, 0.25in) node[anchor = east] {\fontsize{9}{9}\selectfont \color{black!40} \thepage}; % page number
	\end{tikzpicture}
}

% add page layout 
\AtBeginShipout{
	\AtBeginShipoutUpperLeft{\pagelayout}
}

% command to add a subheading
\newcommand{\subheading}[1]{
\vspace{24pt}
{\color{themecolor} \fontsize{14}{14}\selectfont \textbf{#1}}
\vspace{6pt}
\dividerline
\vspace{-20pt}
}

%--------------------------------------------------%
% start document
%--------------------------------------------------%

\begin{document}

\clearpage
\pagestyle{empty}

\color{black!75}

\small

\begin{flushleft}

%--------------------------------------------------%
% cover
%--------------------------------------------------%

\cover{The Czech Constitutional Court Dataset \\ The CCC Dataset}{1.0}{Štěpán Paulík}{16}

\newpage

%--------------------------------------------------%
% table of contents
%--------------------------------------------------%

% reset page counter
\setcounter{page}{1}

% format the table of contents header
% \renewcommand\contentsname{{\color{themecolor} \fontsize{14}{14}\selectfont Datasets}}
\renewcommand\contentsname{\subheading{Datasets} \vspace{0pt}}

% add the table of contents
\tableofcontents

% remove page number from table of contents pages
\addtocontents{toc}{\protect\thispagestyle{empty}}

\newpage

%--------------------------------------------------%
% content
%--------------------------------------------------%


%--------------------------------------------------%
% dataset
%--------------------------------------------------%

\headerpage{ccc\_metadata}{decision-level overview-metadata}{30}{10}

\subheading{Description}

This dataset includes overview information on all 93826 decisions of the CCC spanning from its founding in 1993 to the end of 2023. There is always one observation per decision. The datasets includes information on judge rapporteur, the deciding body, date of the submission and date of decision, as well as nested all the remaining datasets. The information contained herein unlocks the potential to wrangle the data to fit anyone's ends, for example, to create time-series data on caseload of different internal bodies of the CCC and alike.

\subheading{Variables}

\begin{description}[labelwidth=130pt, leftmargin=\dimexpr\labelwidth+\labelsep\relax, font=\normalfont, itemsep=10pt]
\end{description}
%--------------------------------------------------%
% dataset
%--------------------------------------------------%

\headerpage{ccc\_texts}{decision-level text-corpus data}{30}{10}

\subheading{Description}

This dataset includes full text corpus of all CCC decisions. This dataset unlocks the potential for various quantitative text analysis or machine learning applications.

\subheading{Variables}

\begin{description}[labelwidth=130pt, leftmargin=\dimexpr\labelwidth+\labelsep\relax, font=\normalfont, itemsep=10pt]
\end{description}
%--------------------------------------------------%
% dataset
%--------------------------------------------------%

\headerpage{ccc\_judges}{judge-level judges background-data}{30}{10}

\subheading{Description}

This dataset includes judge-level information on all judges that have served at the CCC. The dataset contains information on their background, such as gender, highest degree, or alma mater. All information has been partly automatically and partly manually coded based on publicly available information.

\subheading{Variables}

\begin{description}[labelwidth=130pt, leftmargin=\dimexpr\labelwidth+\labelsep\relax, font=\normalfont, itemsep=10pt]
\end{description}
%--------------------------------------------------%
% dataset
%--------------------------------------------------%

\headerpage{ccc\_clerks}{clerk-level clerks background-data}{30}{10}

\subheading{Description}

This dataset includes judge-level information on all clerks that have served under a judge at the CCC. The dataset contains information on their background, such as gender, highest degree, or whether they studied abroad. All information has been partly automatically and partly manually coded based on publicly available information.

\subheading{Variables}

\begin{description}[labelwidth=130pt, leftmargin=\dimexpr\labelwidth+\labelsep\relax, font=\normalfont, itemsep=10pt]
\end{description}
%--------------------------------------------------%
% dataset
%--------------------------------------------------%

\headerpage{ccc\_compositions}{decision-level composition data}{30}{10}

\subheading{Description}

This dataset contains information on which judges sat at which case. Each observation is a composite obervation of a decision and a judge that took part in that decision. The data has been mined automatically from the text corpus using regular expressions. Because of a rather high degree of irregularity of the decisions from the first decade, the data is not entirely reliable for that period.

\subheading{Variables}

\begin{description}[labelwidth=130pt, leftmargin=\dimexpr\labelwidth+\labelsep\relax, font=\normalfont, itemsep=10pt]
\end{description}
%--------------------------------------------------%
% dataset
%--------------------------------------------------%

\headerpage{ccc\_separate\_opinions}{decision-judge-level separate-opinion data}{30}{10}

\subheading{Description}

This dataset contains information on which judges attached a separate opinion to a decision. Each observation is a composite obervation of a decision and a judge that attached a separate in that decision. The information on whether a judge attached a separate opinion or not comes from the CCC's own database, the remaining data has been to some extent mined automatically from the text corpus using regular expressions. Because of a rather high degree of irregularity of the decisions from the first decade, the data is not entirely reliable for that period.

\subheading{Variables}

\begin{description}[labelwidth=130pt, leftmargin=\dimexpr\labelwidth+\labelsep\relax, font=\normalfont, itemsep=10pt]
\end{description}
%--------------------------------------------------%
% dataset
%--------------------------------------------------%

\headerpage{ccc\_parties}{decision-party level parties data}{30}{10}

\subheading{Description}

This dataset contains the information on parties, both the applicant and the involved party, as well as additional information on the type of the party. Each observation is a decision-party composite.

\subheading{Variables}

\begin{description}[labelwidth=130pt, leftmargin=\dimexpr\labelwidth+\labelsep\relax, font=\normalfont, itemsep=10pt]
\end{description}
%--------------------------------------------------%
% dataset
%--------------------------------------------------%

\headerpage{ccc\_references}{decision-reference-level references acts data}{30}{10}

\subheading{Description}

This dataset contains the information on concerned legal acts, that is both ordinary statutes as well as constitutional acts, and references to CCC caselaw. The information on the concerned acts come from the CCC itself, whereas the references to the CCC caselaw have been automatically mined from the text corpus using regular expressions.

\subheading{Variables}

\begin{description}[labelwidth=130pt, leftmargin=\dimexpr\labelwidth+\labelsep\relax, font=\normalfont, itemsep=10pt]
\end{description}
%--------------------------------------------------%
% dataset
%--------------------------------------------------%

\headerpage{ccc\_subject\_matter}{decision-subject-matter level data}{30}{10}

\subheading{Description}

This dataset contains information on the subject matter of cases. Each observation is a decision and one subject matter composite. The data was created by merging two information from the CCC database, namely on the subject proceedings and the concerned area of constitutional law.

\subheading{Variables}

\begin{description}[labelwidth=130pt, leftmargin=\dimexpr\labelwidth+\labelsep\relax, font=\normalfont, itemsep=10pt]
\end{description}
%--------------------------------------------------%
% dataset
%--------------------------------------------------%

\headerpage{ccc\_disputed\_acts}{decision-disputed-acts-level data}{30}{10}

\subheading{Description}

This dataset includes the disputed acts in any given decision. Each observation is a decision-disputed act composite. The data was obtained from the CCC database. Additionaly, information on the type of the disputed act was automatically added.

\subheading{Variables}

\begin{description}[labelwidth=130pt, leftmargin=\dimexpr\labelwidth+\labelsep\relax, font=\normalfont, itemsep=10pt]
\end{description}
%--------------------------------------------------%
% dataset
%--------------------------------------------------%

\headerpage{ccc\_verdicts}{decision-verdict-level data}{30}{10}

\subheading{Description}

This dataset includes information on the type of verdict the CCC reached in any given decision, i.e., whether the case was granted, rejected, rejected on inadmissibility, or whether it was a procedural verdict. Each observation is a decision-verdict composita. Each decision may contain more than one verdicts.

\subheading{Variables}

\begin{description}[labelwidth=130pt, leftmargin=\dimexpr\labelwidth+\labelsep\relax, font=\normalfont, itemsep=10pt]
\end{description}

%--------------------------------------------------%
% end document
%--------------------------------------------------%

\end{flushleft}

\end{document}
