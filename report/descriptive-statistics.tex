% Options for packages loaded elsewhere
\PassOptionsToPackage{unicode}{hyperref}
\PassOptionsToPackage{hyphens}{url}
%
\documentclass[
]{article}
\usepackage{amsmath,amssymb}
\usepackage{lmodern}
\usepackage{iftex}
\ifPDFTeX
  \usepackage[T1]{fontenc}
  \usepackage[utf8]{inputenc}
  \usepackage{textcomp} % provide euro and other symbols
\else % if luatex or xetex
  \usepackage{unicode-math}
  \defaultfontfeatures{Scale=MatchLowercase}
  \defaultfontfeatures[\rmfamily]{Ligatures=TeX,Scale=1}
\fi
% Use upquote if available, for straight quotes in verbatim environments
\IfFileExists{upquote.sty}{\usepackage{upquote}}{}
\IfFileExists{microtype.sty}{% use microtype if available
  \usepackage[]{microtype}
  \UseMicrotypeSet[protrusion]{basicmath} % disable protrusion for tt fonts
}{}
\makeatletter
\@ifundefined{KOMAClassName}{% if non-KOMA class
  \IfFileExists{parskip.sty}{%
    \usepackage{parskip}
  }{% else
    \setlength{\parindent}{0pt}
    \setlength{\parskip}{6pt plus 2pt minus 1pt}}
}{% if KOMA class
  \KOMAoptions{parskip=half}}
\makeatother
\usepackage{xcolor}
\usepackage[margin=1in]{geometry}
\usepackage{graphicx}
\makeatletter
\def\maxwidth{\ifdim\Gin@nat@width>\linewidth\linewidth\else\Gin@nat@width\fi}
\def\maxheight{\ifdim\Gin@nat@height>\textheight\textheight\else\Gin@nat@height\fi}
\makeatother
% Scale images if necessary, so that they will not overflow the page
% margins by default, and it is still possible to overwrite the defaults
% using explicit options in \includegraphics[width, height, ...]{}
\setkeys{Gin}{width=\maxwidth,height=\maxheight,keepaspectratio}
% Set default figure placement to htbp
\makeatletter
\def\fps@figure{htbp}
\makeatother
\setlength{\emergencystretch}{3em} % prevent overfull lines
\providecommand{\tightlist}{%
  \setlength{\itemsep}{0pt}\setlength{\parskip}{0pt}}
\setcounter{secnumdepth}{-\maxdimen} % remove section numbering
\ifLuaTeX
  \usepackage{selnolig}  % disable illegal ligatures
\fi
\IfFileExists{bookmark.sty}{\usepackage{bookmark}}{\usepackage{hyperref}}
\IfFileExists{xurl.sty}{\usepackage{xurl}}{} % add URL line breaks if available
\urlstyle{same} % disable monospaced font for URLs
\hypersetup{
  pdftitle={Europeanisation of National Constitutional Courts Beyond Extremes: A Few Insights from Czechia},
  pdfauthor={Štěpán Paulík},
  hidelinks,
  pdfcreator={LaTeX via pandoc}}

\title{Europeanisation of National Constitutional Courts Beyond
Extremes: A Few Insights from Czechia}
\author{Štěpán Paulík}
\date{2023-01-20}

\begin{document}
\maketitle

\hypertarget{descriptive-statistics}{%
\subsection{Descriptive Statistics}\label{descriptive-statistics}}

The goal of the descriptive statistics chapter is to build an intuition
regarding what theories about the CC's role face to face the ECJ may be
plausible and to build a more comprehensive picture. The goal is not
inference about whether certain hypotheses hold nor not, nor is it to
say that other theories are wrong.

The data are result of a connection between two datasets: the Iuropa
dataset for ECJ and our own Czech Apex Courts dataset. Both datasets
contain complete text corpora as well as various metadata about the
decisions.

\hypertarget{varying-judge-rapporteur-behavior}{%
\subsubsection{Varying judge rapporteur
behavior}\label{varying-judge-rapporteur-behavior}}

Our first intuition is that it may not be helpful to look at a court as
a whole but to also inquire whether individual judges' usage of ECJ
case-law differs. At first glance, the usage of ECJ case-law differs
across judges even when their yearly caseload is taken into account.
Thus, the second plot shows the relative frequency in percent in how
many cases does a judge refer to an ECJ decision relative to their total
yearly caseload. We can see that some judges stay well bellow 5 \% for
their whole tenure, whereas other judges refer in more than double the
amount. What's more, the development within individual judges is not
static either.

\hypertarget{the-purpose-of-the-ecj-reference}{%
\subsubsection{The purpose of the ECJ
reference}\label{the-purpose-of-the-ecj-reference}}

National constitutional courts within EU are often painted as defendants
of the national constitutionality face to face the competing claim for
authority from the ECJ. The conflict between the ECJ and the
constitutional courts often takes the spotlight in legal literature. Is
this perspective justified by the overall picture of our case study?

At first, it may seem like yes, that's indeed the case. The data is
subset so that only cases that include a reference to an ECJ decision
are included. Quick glance at the overtime development of the ratio
between cases that the CC grants to the applicant against the government
and cases that the CC rejects seems to show nothing irregular - the ECJ
is mostly referred to cases in which the constitution is given
precedence. The ratio between concrete and abstract review is similar -
concrete review cases dominate.

What's noticeable though is the growth of the CC's usage of ECJ caselaw.
The CC is overtime becoming more familiar and europenized the before.
The picture remains stable even if the data are transformed to relative
frequency against the total number of cases that the CC decides so that
the growth of the ``europenization'' doesn't merely reflect the yearly
growth of CC's caseload.

\begin{verbatim}
## `summarise()` has grouped output by 'outcome'. You can override using the
## `.groups` argument.
## `summarise()` has grouped output by 'outcome'. You can override using the
## `.groups` argument.
## Joining, by = c("outcome", "year_cc")
## Joining, by = "year_cc"
\end{verbatim}

\includegraphics{Descriptive-Statistics_files/figure-latex/ratios_absolute-1.pdf}
\includegraphics{Descriptive-Statistics_files/figure-latex/ratios_absolute-2.pdf}

\begin{verbatim}
## `summarise()` has grouped output by 'type_proceedings'. You can override using
## the `.groups` argument.
## `summarise()` has grouped output by 'type_proceedings'. You can override using
## the `.groups` argument.
## Joining, by = c("type_proceedings", "year_cc")
## Joining, by = "year_cc"
\end{verbatim}

\begin{verbatim}
## Warning in remove(matches_joined_type_proceedings_freq_relative,
## matches_joined_type_proceedings_freq_total): object
## 'matches_joined_type_proceedings_freq_total' not found
\end{verbatim}

\includegraphics{Descriptive-Statistics_files/figure-latex/ratios_absolute-3.pdf}
\includegraphics{Descriptive-Statistics_files/figure-latex/ratios_absolute-4.pdf}

However, the results get much more interesting when we focus on the
relative frequency of the amount of granted decisions/abstract review
decisions per year divided by the total number of that type of case of
that year. That is to account for not only variation of the CC's
case-load over time but to also account for variation within the years.
\includegraphics{Descriptive-Statistics_files/figure-latex/analysis-1.pdf}
\includegraphics{Descriptive-Statistics_files/figure-latex/analysis-2.pdf}

\end{document}
